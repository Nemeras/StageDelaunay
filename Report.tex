\documentclass[a4paper,10pt]{article}
\usepackage[english]{babel}
\usepackage[utf8]{inputenc}
\usepackage[margin=1.5in]{geometry}
\usepackage{amsmath}
\usepackage{amsthm}
\usepackage{amsfonts}
\usepackage{amssymb}
\usepackage[usenames,dvipsnames]{xcolor}
\usepackage{graphicx}
\usepackage[siunitx]{circuitikz}
\usepackage{tikz}
\usepackage{hyperref}
\usepackage[numbers, square]{natbib}
\usepackage{fancybox}
\usepackage{epsfig}
\usepackage{soul}
\usepackage[framemethod=tikz]{mdframed}
\usepackage[shortlabels]{enumitem}
\usepackage[version=4]{mhchem}
\usepackage{fullpage}

%opening

\title{
\normalfont \normalsize 
\textsc{ENS Lyon} \\
[10pt] 
\rule{\linewidth}{0.5pt} \\[6pt] 
\huge Formalisation of the Delaunay Triangulation \\
\rule{\linewidth}{2pt}  \\[10pt]
}
\author{Clément Sartori}
\date{\normalsize 16/06/2017}

\begin{document}

\maketitle
\noindent
%Date Performed \dotfill December 31, 1999 \\
%Partners \dotfill Full Name \\
Advisor \dotfill Yves Bertot \\
%\title{Formalisation of the Delaunay triangulation}
%\author{Clément Sartori}


\maketitle

\begin{abstract}
  
\end{abstract}

\section{Introduction}
\rule{\linewidth}{0.5pt}
\subsection{Coq/The Mathematical Component library}

\subsection{Triangulations and Applications}

\section{The Algorithm}
\rule{\linewidth}{0.5pt}

\section{Geometrical Properties}
\rule{\linewidth}{0.5pt}

\section{Instanciation of the model}
\rule{\linewidth}{0.5pt}


\section{Examples of proofs}
\rule{\linewidth}{0.5pt}




\end{document}


\end{document}
