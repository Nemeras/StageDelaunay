\documentclass[handout]{beamer}

\usepackage[utf8]{inputenc}
\usepackage{default}
\usepackage{graphicx}
\usepackage{amsmath}
\usepackage{amsthm}
\usepackage{amsfonts}
\usepackage{amssymb}
\usefonttheme{professionalfonts}

\usetheme{Warsaw}

\beamertemplatenavigationsymbolsempty
\addtobeamertemplate{navigation symbols}{}{%
    \usebeamerfont{footline}%
    \usebeamercolor[fg]{footline}%
    \hspace{1em}%
    \insertframenumber/\inserttotalframenumber
}

\useoutertheme{smoothbars}
\useinnertheme[shadow=true]{rounded}
\usecolortheme{whale}

%\newtheorem{definition}{Definition}
\begin{document}

\title{Formalisation of the Delaunay triangulation}
\author{Clément Sartori}
\begin{frame}
\maketitle
\end{frame}
\begin{frame}
 \tableofcontents
\end{frame}

\section{Introduction}

\begin{frame}
 Goals of the internship :
 \begin{enumerate}
  \item<1,2> Formalise the Delaunay triangulations.
  \item<2> Study some of the processes involved in constructing a Delaunay triangulation.
 \end{enumerate}

 
\end{frame}

\subsection{Triangulations}
\begin{frame}
 Delaunay triangulations have lots of applications :
 \begin{itemize}
  \item<1,2,3> Shape morphing.
  \item<2,3> Modelise cellular coverage maps.
  \item<3> Path planning.
 \end{itemize}
 
 We anticipate the use of these algorithms in mission-critical software for autonomous vehicles: it is interesting to try and formalise Delaunay triangulations.
 
 We tried to continue a previous work \cite{Bertot} by Yves Bertot and Jean-Francois Dufourd, but in a more abstract way.

\end{frame}

\begin{frame}
 
 A triangulation $T$ of a subset $X$ of $\mathbb{R}^d$ is a tiling of this subset with simplices such
that:
\begin{itemize}
 \item<1,2,3> the intersection of two distinct simplices is either empty or a common face of these simplices;
 \item<2,3> every bounded set of R d cuts a finite number of simplices of T 2 ;
 \item<3> the union of the simplices of T is X itself.
\end{itemize}


\end{frame}


\subsection{{\sc Coq}}

\begin{frame}
 
\end{frame}


\subsection{CC Systems}

\begin{frame}
 
\end{frame}


\section{The Algorithm}

\subsection{Adding Points}

\begin{frame}
 
\end{frame}


\subsection{Flipping Edges}

\begin{frame}
 
\end{frame}


\section{The Formalisation}

\subsection{Functions}

\begin{frame}
 
\end{frame}

\subsection{Geometrical Predicates}

\begin{frame}
 
\end{frame}


\subsection{Hypotheses}

\begin{frame}
 
\end{frame}


\subsection{Geometrical Properties}

\begin{frame}
 
\end{frame}


\section{Operations and Proofs}

\begin{frame}
 
\end{frame}

\section{Conclusion}

\begin{frame}
\bibliographystyle{alpha}
\bibliography{Report} 
\end{frame}




\end{document}
